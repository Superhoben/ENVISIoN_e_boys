\section{How to build Inviwo with ENVISIoN on Ubuntu Linux 20.04}

\label{ch:install}
These instructions show how to build Inviwo and ENVISIoN on Ubuntu Linux 20.04.

\subsection{Install git}
Start by installing git, which will be used to fetch ENVISIoN in the
next step.
\begin{lstlisting}[frame = single, breaklines=true]
    sudo apt install git
\end{lstlisting}

\subsection{Install dependencies}
Install the dependencies for ENVISIoN: 
\begin{lstlisting}[frame = single, breaklines = true]
  sudo apt install \
    python3 python3-pip \
    git \
    python3-numpy python3-h5py python3-pybind11 \
    python3-scipy python3-regex \
    npm
\end{lstlisting}

Install the dependencies for building Inviwo: 
\begin{lstlisting}[frame = single, breaklines = true]
    sudo apt install \
      build-essential gcc-8 g++-8 cmake git freeglut3-dev xorg-dev \
      openexr zlib1g zlib1g-dev \
      qt5-default qttools5-dev qttools5-dev-tools \
      python3 python3-pip \
      libjpeg-dev libtiff-dev libqt5svg5-dev libtirpc-dev libhdf5-dev
      libpng-dev libglu1-mesa-dev libxrandr-dev \
      libxinerama-dev libxcursor-dev
\end{lstlisting}

Check that you have access to Qt 5.12 or later: 
\begin{lstlisting}[frame = single, breaklines = true]
    qmake --version
\end{lstlisting}

If you need to upgrade Qt, you can do the following:
\begin{lstlisting}[frame = single, breaklines = true]
    wget http://download.qt.io/official_releases/qt/5.12/5.12.2/qt-opensource-linux-x64-5.12.2.run
    chmod +x qt-opensource-linux-x64-5.12.2.run
    sudo ./qt-opensource-linux-x64-5.12.2.run
    qtchooser -install opt-qt5.12.2 /opt/Qt5.12.2/5.12.2/gcc_64/bin/qmake
\end{lstlisting}

Check that you have access to cmake 3.12.0 or later by the following command:
\begin{lstlisting}[frame = single, breaklines = true]
    cmake --version
\end{lstlisting}

\subsection{Setup ENVISIoN}
You may build ENVISIoN in whatever directory you want. However, this guide will assume that you build it under your home directory. 

Create a directory under your home directory to build ENVISIoN:

\begin{lstlisting}[frame = single, breaklines=true]
    mkdir ~/ENVISIoN
    cd ~/ENVISIoN
\end{lstlisting}

Download ENVISIoN and install the electron-based gui dependencies:

\begin{lstlisting}[frame = single, breaklines=true]
    git clone https://github.com/rartino/ENVISIoN
    cd ENVISIoN
    npm install
\end{lstlisting}

\subsection{Build Inviwo}
Download and checkout the correct version of the Inviwo source:

\begin{lstlisting}[frame = single, breaklines=true]
    cd ~/ENVISIoN
    git clone https://github.com/inviwo/inviwo
    cd inviwo
    git checkout v0.9.11
\end{lstlisting}

Install the Inviwo submodule dependencies:

\begin{lstlisting}[frame = single, breaklines=true]
    sed -i 's%https://github.com/live-clones/hdf5.git%https://github.com/HDFGroup/hdf5.git%' .gitmodules
    git submodule update --init --recursive
\end{lstlisting}

Apply the ENVISIoN patches to Inviwo:

\begin{lstlisting}[frame = single, breaklines=true]
    git apply \
        "$HOME/ENVISIoN/ENVISIoN/inviwo/patches/deppack_fix.patch" \
        "$HOME/ENVISIoN/ENVISIoN/inviwo/patches/filesystem_env.patch" \
        "$HOME/ENVISIoN/ENVISIoN/inviwo/patches/ftl_fix.patch" \
        "$HOME/ENVISIoN/ENVISIoN/inviwo/patches/transferfunction_extras.patch"
\end{lstlisting}
\newpage
\subsection{Cmake build using system compilers}
Choose the correct Qt version by the following commands:

\begin{lstlisting}[frame = single, breaklines=true]
    qtchooser -l
    export QT_SELECT=<qt version>
    eval `qtchooser --print-env`
\end{lstlisting}

Configure and build Inviwo:

\begin{lstlisting}[frame = single, breaklines=true]
    cd ~/ENVISIoN
    mkdir inviwo-build
    cd inviwo-build/
    cmake -G "Unix Makefiles" \
      -DCMAKE_C_COMPILER="gcc-8" \
      -DCMAKE_CXX_COMPILER="g++-8" \
      -DBUILD_SHARED_LIBS=ON \
      -DIVW_USE_EXTERNAL_IMG=ON \
      -DIVW_EXTERNAL_MODULES="$HOME/ENVISIoN/ENVISIoN/inviwo/modules" \
      -DIVW_MODULE_CRYSTALVISUALIZATION=ON \
      -DIVW_MODULE_GRAPH2D=ON \
      -DIVW_MODULE_HDF5=ON \
      -DIVW_USE_EXTERNAL_HDF5=ON \
      -DIVW_MODULE_PYTHON3=ON \
      -DIVW_MODULE_PYTHON3QT=ON \
      -DIVW_MODULE_QTWIDGETS=ON \
      -DIVW_PACKAGE_PROJECT=ON \
      -DIVW_PACKAGE_INSTALLER=ON \
      -S ../inviwo -B ./
    make -j4
\end{lstlisting}

\emph{Note:}
\begin{itemize}
     \item The number in make -j4 is the number of simultaneous build processes to run. Usually the best choice is the number of CPU cores in your build system.
     \item If you are running into build errors, re-run make with make -j1 to make sure that the last printout pertains to the actual error.
\end{itemize}

\subsection{Verify that the build works correctly}
Start Inviwo to make sure it is built correctly:

\begin{lstlisting}[frame = single, breaklines=true]
    cd ~/ENVISIoN
    inviwo-build/bin/inviwo
\end{lstlisting}

Quit Inviwo and start the ENVISIoN GUI to see that it works:

\begin{lstlisting}[frame = single, breaklines=true]
    cd ~/ENVISIoN/ENVISIoN
    export INVIWO_HOME="$HOME/ENVISIoN/inviwo-build/bin"
    npm start
\end{lstlisting}

\newpage